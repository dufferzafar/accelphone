\begin{center}
\LARGE{Abstract}
\end{center}

\vspace{0.5in}

The popularity of smartphones continues to grow because of the wide variety of functionality they offer - from just being able to make calls and access the internet to recording videos and playing games. To provide a lot of this functionality, a modern smartphone comes equipped with sensors such as camera, microphone, GPS etc. Data from these sensors enables the creation of applications that offer rich and personalised user experience.

Use of these sensors also opens up the possibilities of new attacks by leaking information via side channels. In this report, we explore how data from a particular mobile sensor - the accelerometer - can be used to eavesdrop acoustic signals in the vicinity of the phone, thereby converting the accelerometer into a microphone, and since accessing the sensor doesn't require any special permission, this allows an adversary unregulated access to audio surrounding the device's environment.
