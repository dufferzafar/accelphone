
\chapter{Introduction}

As mobile phones become more and more ubiquitous, they are no longer used just for communication purposes but also as personalized computing devices, since a smartphone offers a mix of functions ranging from telephony and internet connectivity to use as a camera.

Like any technology, mobile phones have their caveats - with their widespread use, the issues of security and privacy have taken a central role in their design and usage. Since the computation platform a modern smartphone offers is akin to a general purpose computer, they suffer from similar threats like viruses and ransomwares and also more targetted attacks like leaking of sensitive information etc.

Smartphones today, come equipped with a wide variety of sensors like camera, microphone, GPS to enable applications that offer a rich user experience. However, they are not without their own caveats - recent research
% TODO: Cite papers
\textbf{[CITE]} has shown a host of different ways how these sensors leak information that can be used against the users and has far reaching privacy implications.

A particular class of sensors is used for motion detection - gyroscopes, accelerometer etc. While privacy issues associated with the use of a microphone and GPS are understood by most users, those related with motion sensors are not. So an adversary is more likely to use such senors.

% TODO: Write a bit more, linking to accelerometer

\section{Research Goal}

% Our problem statement can be defined as follows:
\begin{adjustwidth}{1cm}{1cm}
The goal of this project is to explore how MEMS accelerometers can be used as microphones and see if they are sensitive enough to eavesdrop audio in the environment surrounding a phone.
\end{adjustwidth}

\newpage

\section{MEMS Accelerometer}
% A paragraph about their working; image?
% \textbf{[CITE]}
Accelerometers in smartphones are based on Micro Electro Mechanical Systems (MEMS) design, which emulate mechanical parts through micro-machining technology. At their core, the MEMS accelerometers have a sensing mass, suspended with springs, which gets displaced due to forces (that cause acceleration.) It was shown in \cite{walnut} that accelerometers are susceptible to acoustic interferences, i.e. an acoustic wave can exert a force on the sensor that is strong enough to displace the sensing mass affecting the sensor's output.

\section{Sensors on an Android Phone}
\label{ch1-intro-android}
% \section{Android's Permission Model}

Android is a mobile operating system from Google. Launched in 2008, it's growth paralleled that of smartphones themselves. It has the majority marketshare today with around 85\% of all smartphones coming with Android installed (as of 2017.) \cite{androidshare}

One of the reasons of Android's growth are its permissive APIs - applications on this platform enjoy a lot more freedom compared to iOS and even though new features (to both hardware and software) are added in a quest to increase the user experience, the security \& privacy preserving aspects take a back seat and do not improve at the same pace.

Applications that require access to a particular hardware capability need to ask permission from the user to be able to use it. This is the core line of defense that prevents malicious applications from being able to abuse critical sources of information like camera, microphone, SMS etc.

However, sensors like accelerometer, gyroscope etc. do not require a special permission and can be accessed by any application, in both foreground and background. This allows an attacker to use data from these zero-permission sensors maliciously without any sort of consent from the user.

% What they're used for etc. Mention the limitations?

% One such problem is that of Android's Permission Model -


% This allows users to


% One problem is that the pool of sensors embedded in smartphones may un- intentionally leak sensitive information. For example, sensors that are accessible without user permissions, so-called zero-permission sensors, may be exploited by an attacker, without the knowledge of the user. The scenario is analogous to other widely known side-channel attacks which exploit unintentional (physi- cal) leakage. For example, this can be used to extract cryptographic keys [7]. In our case, the unintentional leakage from sensors reveals privacy-related informa- tion about the user, such as personal identification number (PIN) or movement patterns.


% Zero permission sensors

\newpage
\section{Recent Developments}

% Also mention the potential fix in Android P

In March 2018, Google announced the first developer preview for the privacy centered features of the next Android version - P. Among the major behaviour changes, there are some privacy centeric changes too, wherein, the applications will not be able to record motion sensors, microphone, camera in background. Applications that legitimately need to record motion sensors will have to show a persistent notification so that the user is aware of what is happening in the background. This serves use cases for apps that use motion data to perform tasks such as step counting etc.

Foreground applications will have no such restrictions and will continue to work the same way.

% It remains to be seen

% A beta version of Android P was released two days ago
