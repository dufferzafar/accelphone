\chapter{Introduction}

As mobile phones become more and more ubiquitous, they are no longer used just for communication purposes but also as personalized computing devices, since a smartphone offers a mix of functions ranging from telephony and internet connectivity to use as a camera.

Like any technology, mobile phones have their caveats - with their widespread use, the issues of security and privacy have taken a central role in their design and usage. Since the computation platform a modern smartphone offers is akin to a general purpose computer, they suffer from similar threats like viruses and ransomwares and also more targetted attacks like leaking of sensitive information etc.

Smartphones today, come equipped with a wide variety of sensors like camera, microphone, GPS to enable applications that offer a rich user experience. However, they are not without their own caveats - recent research
% TODO: Cite papers
\textbf{[CITE]} has shown a host of different ways how these sensors leak information that can be used against the users and has far reaching privacy implications.

A particular class of sensors is used for motion detection - gyroscopes, accelerometer etc. While privacy issues associated with the use of a microphone and GPS are understood by most users, those related with motion sensors are not. So an adversary is more likely to use such senors.

% TODO: Write a bit more, linking to accelerometer

\section{Research Goal}

% Our problem statement can be defined as follows:
\begin{adjustwidth}{1cm}{1cm}
The goal of this project is to explore how accelerometer can be used as a microphone.
\end{adjustwidth}

\section{Working of MEMS Sensors}
A paragraph about their working; image?

% \section{Sensors on Mobile Phones}
% What they're used for etc. Mention the limitations?

\section{Android's Permission Model}
Also mention the potential fix in Android P

\lipsum[2]

